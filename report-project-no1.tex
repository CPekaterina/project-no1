\documentclass[11pt,a4wide]{article}
\usepackage{verbatim}
\usepackage{listings}
\usepackage{graphicx}
\usepackage{a4wide}
\usepackage{color}
\usepackage{amsmath}
\usepackage{amssymb}
\usepackage[dvips]{epsfig}
\usepackage[ansinew]{fontenc}
\usepackage{cite} % [2,3,4] --> [2--4]
\usepackage{shadow}
\usepackage{hyperref}

\setcounter{tocdepth}{2}
%
\lstset{language=c++}
\lstset{alsolanguage=[90]Fortran}
\lstset{basicstyle=\small}
\lstset{backgroundcolor=\color{white}}
\lstset{frame=single}
\lstset{stringstyle=\ttfamily}
\lstset{keywordstyle=\color{red}\bfseries}
\lstset{commentstyle=\itshape\color{blue}}
\lstset{showspaces=false}
\lstset{showstringspaces=false}
\lstset{showtabs=false}
\lstset{breaklines}

\begin{document}
\tableofcontents
\newpage
\section*{The algorithm}
In order to solve the one-dimensional Poisson equation
\begin{equation}
-u''(x)=f(x)
\label{eq:1}
\end{equation}
with Dirichlet boundary conditions in the interval $(0,1)$ we rewrite the latter as a set of linear equations by discretizing the problem. In this way we obtain a set of $n$ grid points with the gridwidth $h=1/(n+1)$.Then we approximate the second derivative $u''(x)$ with
\begin{equation}
-\dfrac{-v_{i+1}-v_{i-1}+2v_i}{h^2}=f_i\qquad \text{for}\; i=1,..,n
\label{eq:2}
\end{equation}
From this we can easily derive the following matrix equation:
\begin{equation}
     \left(\begin{array}{cccccc}
                           2& -1& 0 &\dots   & \dots &0 \\
                           -1 & 2 & -1 &0 &\dots &\dots \\
                           0&-1 &2 & -1 & 0 & \dots \\
                           & \dots   & \dots &\dots   &\dots & \dots \\
                           0&\dots   &  &-1 &2& -1 \\
                           0&\dots    &  & 0  &-1 & 2 \\
                      \end{array} \right)\left(\begin{array}{c}
                           v_0\\
                           v_1\\
                           \dots \\
                          \dots  \\
                          \dots \\
                           v_{n-1}\\
                      \end{array} \right)
  =\left(\begin{array}{c}
                           h^2f_0\\
                           h^2f_1\\
                           \dots \\
                           \dots \\
                          \dots \\
                           h^2f_{n-1}\\
                      \end{array} \right).
\label{N}\end{equation}
A more general form of the above is:
\begin{equation}
    \left(\begin{array}{cccccc}
                           a_0& b_0 & 0 &\dots   & \dots &\dots \\
                           c_1 & a_1 & b_1 &\dots &\dots &\dots \\
                           & c_2 & a_2 & b_2 & \dots & \dots \\
                           & \dots   & \dots &\dots   &\dots & \dots \\
                           &   &  &c_{n-2}  &a_{n-2}& b_{n-2} \\
                           &    &  &   &c_{n-1} & b_ {n-1}\\
                      \end{array} \right)\left(\begin{array}{c}
                           v_0\\
                           v_1\\
                           \dots \\
                          \dots  \\
                          \dots \\
                           v_{n-1}\\
                      \end{array} \right)
  =\left(\begin{array}{c}
                           w_0\\
                           w_1\\
                           \dots \\
                           \dots \\
                          \dots \\
                           w_{n-1}\\
                      \end{array} \right).
											\label{M}
\end{equation}
Since the Gaussian elimination would of course lead to the correct results here, the execution time can be easily reduced from $\sim n^3$ to $\sim n$ by applying an algorithm that no longer requires the matrix but uses the three diagonals as arrays. In other words we consider that the rest of the matrix is 0 everywhere except for these diagonals which the brute force Gaussian elimination way does not take in account. The following steps have then to be taken:
\begin{enumerate}
	\item  The three diagonals are stored in arrays $a[],\,b[],$ and $c[]$, as well as the right side of the equation is stored in an array $w[]$ of the size $n$. $c[0]$ and $b[n-1]$ are set to 0.
	\item	Then the entries in $a[]$ are substituted recursively by
	\begin{equation}
	\tilde{a}[0]=a[0],\quad \tilde{a}[i]=a[i]-b[i-1]\dfrac{c[i]}{\tilde{a}[i-1]}
	\label{eq:3}
	\end{equation}
	This requires $3\cdot (n-1)$ floating point operations for we obtain a division, a substraction and a multiplication for each substitution.
	\item Accordingly $w[]$ is substituted by
		\begin{equation}
	\tilde{w}[0]=w[0],\quad \tilde{w}[i]=w[i]-\tilde{w}[i-1]\dfrac{c[i]}{\tilde{a}[i]}
	\label{eq:4}
	\end{equation}
	This only requires $2\cdot (n-1)$ flops for we already did the division $\frac{c[i]}{\tilde{a}[i-1]}$ during the substitution above.
	\item Finally backward substitution is used to gain the result for the unknown vector $v$ which is stored in another array $v[]$:
		\begin{equation}
	v[n-1]=\dfrac{\tilde{w}[n-1]}{\tilde{a}[n-1]},\quad v[i]=\dfrac{\tilde{w}[i]-b[i]\cdot v[i+1]}{\tilde{a}[i-1]}
	\label{eq:5}
	\end{equation}
	This operation results in another $3\cdot(n-1)+1$ flops for we have again a substraction, a multiplication and a division for each resubstitution plus a division for the first element.
\end{enumerate}
In sum the algorithm needs $8\cdot(n-1)+1$ floating point operations to solve the general matrix equation \ref{M}. 
If we now go back to the specific marix \ref{N} we can simplify the algorithm once more by tuning the above one to our needs.  For this we go through the previous algorithm and insert the given values for the entries of the vectors $a[],\;b[]$ and $c[]$:
\begin{enumerate}
	\item The entries in the diagonals now are:
	\begin{equation}
	a[i]=2\quad \text{for}\; i=0,..,n-1,\quad	b[i]=-1\quad \text{for}\; i=0,..,n-2,\quad	c[i]=-1\quad \text{for}\; i=1,..,n-1
	\label{eq:6}
	\end{equation}
	In this way the array computed in \ref{eq:3} becomes a static vector that can be computed recursively once and then stored somewhere for it does not depend on the right side of the equation:
	\begin{equation}
	\tilde{a}[0]=a[0],\quad \tilde{a}[i]=a[i]-b[i-1]\dfrac{c[i]}{\tilde{a}[i-1]}=a[i]-\dfrac{1}{a[i-1]}
	\label{eq:7}
	\end{equation}
	Therefore this does require $2\cdot (n-1)$ flops -  but only once, so we can compute $\tilde{a}[]$ once for a very large $n$ and then make use of it for any given $w[]$ 
	\item Latter then is to be substituted as follows
	\begin{equation}
	\tilde{w}[0]=w[0],\quad \tilde{w}[i]=w[i]-\tilde{w}[i-1]\dfrac{c[i]}{\tilde{a}[i]}=	w[i]+\dfrac{\tilde{w}[i-1]}{\tilde{a}[i]}
	\label{eq:4}
		\end{equation}
	This requires $2\cdot(n-1)$ floating point operations.
	\item The backward substitution can then be simplified, too:
		\begin{equation}
	v[n-1]=\dfrac{\tilde{w}[n-1]}{\tilde{a}[n-1]},\quad v[i]=\dfrac{\tilde{w}[i]-b[i]\cdot v[i+1]}{\tilde{a}[i]}=\dfrac{\tilde{w}[i]+v[i+1]}{\tilde{a}[i]}
	\label{eq:5}
	\end{equation}
	This operation leads to another $2\cdot(n-1)+1$ flops.
\end{enumerate}
In sum we obtain $4\cdot(n-1)+1$ floating point operations which is more than twice as fast as the general algorithm for tridiagonal matrices.
\\
Possible implementations of both algorithms are shown below.
\begin{lstlisting}[title={solvers for tridiagonal matrices}]
double* tridiagonal (double*a, double*b, double*c, double*w, int n) 	//solves the general tridiagonal matrix
{
    double* v;
    v = new double[n];

    for (int i=1; i<n; i++)
    {
        double temp = c[i]/a[i-1];	//saves n-1 flops
        a[i] -= b[i-1]*temp;		//1.
        w[i] -= w[i-1]*temp;		//2.
    }

    v[n-1] = w[n-1]/a[n-1];		//3.		

    for (int i=n-2; i>=0; i--)
    {
        v[i] = (w[i]-b[i]*x[i+1])/a[i];
    }
    return v;
}

double* tridiagonaldiff (double* a, double* w, int n)	
//solves the discretized Poisson equation, here *a is a pointer to the static array (see 1.)
 {
    double* v;
    v = new double[n];

    for (int i=1; i<n; i++)
    {
        w[i] += w[i-1]/a[i-1];		//2.
    }

    v[n-1] = w[n-1]/a[n-1];		//3.

    for (int i = n-2; i >= 0; i--)
    {
        v[i] = (w[i]+v[i+1])/a[i];
    }
    return v;

}
\end{lstlisting}
\end{document}

